
\documentclass[10pt]{beamer}

\usepackage{beamerthemesplit}
\usepackage{amsmath}
\usepackage{amssymb,mathpartir}
\usepackage{amsthm,listings,cite}
\usepackage{graphicx,multirow}
%\usepackage[latin1]{inputenc}

\useoutertheme{infolines}

\title{How to Write a Trustworthy Theorem Prover}
\subtitle{A Discussion of Jared Davis' Thesis Work on Milawa}

\author{Dionna Glaze}

\date{April 14th, 2010}

\definecolor{blue}{rgb}{0,0,1.0}
\newcommand{\blue}[1]{{\color{blue}#1}}
\definecolor{green}{rgb}{0,1.0,0}
\newcommand{\green}[1]{{\color{green}#1}}
\definecolor{red}{rgb}{1.0,0,0}
\newcommand{\red}[1]{{\color{red}#1}}
\definecolor{grey}{rgb}{0.6,0.6,0.6}
\newcommand{\grey}[1]{{\color{grey}#1}}

\AtBeginSection[]
{
\begin{frame}<beamer>
  \frametitle{Outline}
  \tableofcontents[currentsection]
\end{frame}
}

\begin{document}

\frame{\titlepage}

\frame{
\frametitle{Outline}
\tableofcontents
}

\section{Introduction}

\frame{
\frametitle{Why Prove Theorems?}
\begin{itemize}
 \item We depend on computers.
 \item Our lives are staked on certain systems not failing. \\
\begin{center}
 \includegraphics[scale=0.20]{airplane_in_flight.jpg} 
\end{center}
 \item Just trusting smart people for correctness is not the best idea.
\pause
 \item Solution: Math.
\end{itemize}
}

\frame{
\frametitle{Theorem Provers Today}

\begin{itemize}
 \item Many theorem provers exist, all depend on trust.
 \item How much trust depends on the theorem prover, but in the case of ACL2, one must be very trusting.
\pause
 \item Stanford LCF was created to address the issue of trust.
\pause
 \item How do we get the speed of ACL2 with the LCF philosophy?
\end{itemize}
}

\frame{
\frametitle{Milawa: A Reflective Proof Checker}

\centering{\includegraphics[scale=0.15]{logo-with-title.png}}
\begin{itemize}
 \item Milawa is an attempt to prove an ACL2-like system correct.
 \item It differs from LCF in that it is first order and is extensible.
\pause
 \item Once proven, extensions do not need to build proofs.
\end{itemize}
}

\section{Defining Proof Succinctly}

\frame{
\frametitle{Math as Humans Do It}

\begin{itemize}
 \item Milawa, ACL2, HOL and others use term rewriting. \\
\pause
e.g. Given an axiom 
\begin{equation*}
x \vee \neg x
\end{equation*}
and a theorem 
\begin{equation*}
f(g(x)) = x
\end{equation*}
we create a rewrite rules \\
\begin{center}
\texttt{(or x (not x))} $\rightarrow$ \texttt{True} \\
\texttt{(f (g x))} $\rightarrow$ \texttt{x}
\end{center} 
\pause
 So in our rewrite system, \\
 \texttt{(or (= (f (g x)) y) (not (= x y)))} rewrites to \\
 \texttt{(or (= x y) (not (= x y)))} and then to \texttt{True}.
\end{itemize}
}

\frame{
\frametitle{Proof is Justification}
\begin{itemize}
 \item A proof is a list of steps that rewrite a formula to True.
 \item A step consists of a method, a conclusion, subproofs and extras.
 \item The core of Milawa has very few methods and are hand-checkable.
\end{itemize}
}

\frame{
\frametitle{Rules of inference, some axioms}
\begin{columns}[2]
  \column{.4\textwidth}
  \renewcommand{\arraystretch}{2}
  \begin{tabular}{p{2.2cm} l}
    Propositional Schema & \scriptsize{$\inferrule{ }{\neg A \vee A}
      $}\\
    Contraction & \scriptsize{$\inferrule{A \vee A}{A}$} \\
    Expansion & \scriptsize{$\inferrule{A}{B \vee A}$} \\
    Associativity & \scriptsize{$\inferrule{A \vee (B \vee C)}{(A \vee B) \vee C}$}
    \\
    Cut & \scriptsize{$\inferrule{A \vee B \\ \neg A \vee C}{B \vee C}$} \\
    Instantiation & \scriptsize{$\inferrule{A}{A/\sigma}$} \\
    Induction & ...
  \end{tabular}

  \column{.55\textwidth}
    Reflexivity Axiom \\
      \quad {\scriptsize $x = x$} \\
    \hspace*{\fill} \\
    Equality Axiom \\
      \quad {\scriptsize $x_1 = y_1 \rightarrow x_2 = y_2 \rightarrow x_1
    = x_2 \rightarrow y_1 = y_2$} \\
    \hspace*{\fill} \\
    Referential Transparency \\
      \quad {\scriptsize $x_1 = y_1 \rightarrow ... \rightarrow
    x_n = y_n \rightarrow f(x_1,...,x_n) = f(y_1, ..., y_n)$} \\
    \hspace*{\fill} \\
    Beta Reduction \\
      \quad {\scriptsize $((\lambda x_1 ... x_n . \beta) t_1 ... t_n) =
    \beta / \left[x_1 \leftarrow t_1, ..., x_n \leftarrow t_n
    \right]$} \\
    \hspace*{\fill} \\
    Base Evaluation \\
      \quad {\scriptsize e.g. $1 + 2 = 3$} \\
    \hspace*{\fill} \\
    Lisp Axioms \\ 
      \quad {\scriptsize e.g. \texttt{(consp (cons x y)) = t}}
\end{columns}
}

\frame{
\frametitle{Huge Proofs}

\begin{itemize}
 \item Using only the core just shown results in ridiculously large proofs of trivial theorems.
 \item One reason for this is the inherent use of sharing in proofs.
\pause
 \item This only checks proofs - it does not find them.
\pause
 \item What does a good theorem prover need to be usable?
\end{itemize}
}

\section{Necessary Gadgetry}

\begin{frame}[fragile]
\frametitle{Backchaining Problems}
\begin{itemize}
 \item Theorem proving without hypotheses is a drag.
 \item To apply a rule with hypotheses, you must prove the hypotheses hold.
\pause
 \item Problem rules when rewriting \texttt{(consp a)}: \\
{\scriptsize\begin{lstlisting}
(implies (consp (cdr x))
         (equal (consp x) t)) 
\end{lstlisting}}
\pause
Two rules that combine nastily:
\pause
{\scriptsize\begin{lstlisting}
(implies (true-listp x) 
         (equal (consp x) (if x t nil))) 

(implies (not (consp x))
         (equal (true-listp x) (not x)))
\end{lstlisting}}
\end{itemize}
\end{frame}

\frame{
\frametitle{Ancestors Checking}
\begin{itemize}
 \item A basic but successful approach is to maintain an \emph{ancestors stack}.
 \item Every backchain adds a frame that records
  \begin{itemize}
   \item term to relieve
   \item rule that caused the backchain
   \item Guts of the term
   \item Number of function occurrences.
  \end{itemize}
\pause
 \item Before backchaining, compare new hypothesis to stack and prohibit if:
  \begin{itemize}
    \item Exact term or its guts exists on stack as term to relieve
    \item New term looks ``worse'' than all others caused by the rule \textbf{and} guts of term and frame terms are application of the same function.
  \end{itemize}
\end{itemize}

}

\begin{frame}[fragile]
\frametitle{Free Variable Matching}
\begin{itemize}
 \item  Many rules such as transitivity results have terms in the hypotheses that do not appear in the conclusion. \\
{\scriptsize\begin{lstlisting}
 (implies (and (subsetp x y)
               (subsetp y z))
          (subsetp x z))
\end{lstlisting}}
 \item To allow the rule to be applied, the hypotheses must be discharged, but how?
\pause
 \item Using the substitution from matching \texttt{(subsetp a c)} we end up having to prove \texttt{(subsetp a y)} and \texttt{(subsetp y c)} which will fail. 
\pause
 \item How do we choose y?
\end{itemize}
\end{frame}

\frame{
\frametitle{Free Variable Matching}
\begin{itemize}
 \item For each free variable $v$, mark the first hypothesis to mention it.
 \item Try all bindings which can be sure to satisfy all marked hypotheses.
 \item Rewriter will produce a trace or say the rule cannot be applied.
 \item Recursively try to relieve hypotheses (backchain).
\end{itemize}
}

\frame{
\frametitle{Syntactic Restrictions}
\begin{itemize}
 \item Associativity and commutativity rules are everywhere.
 \item Systems like Maude and Coq treat these specially.
 \item ACL2 and Milawa: permutative rules only apply if conclusion is ``smaller.''
\pause
 \item Can also restrict rules to syntax patterns. e.g. left associative constant folding with right associative normalizing rules.
\pause
 \item Implement by constructing ground substitutions (quote terms) and applying to restrictions. Must evaluate to non-nil.
\end{itemize}

}

\frame{
\frametitle{Rewriter Caching}
\begin{itemize}
 \item Many terms are repeated during proofs.
 \item Rewriting them each time is not desirable.
 \item Use fast association lists to memoize.
\pause
 \item Maintaining the soundness proof means caches should be constructable with a fixed set of assumptions.
 \item This means using fresh caches when rewriting \texttt{if} branches.
\pause
 \item Ignore rewrite limit in memoization.
 \item Do not ignore backchain limit (fail to relieve \texttt{(consp x)} with low backchain limit?)
\end{itemize}
}

\frame{
\frametitle{Rewriting Clauses}
\begin{itemize}
 \item Milawa works on formulas in CNF.
 \item To rewrite clauses, force assumptions that all other terms are false.
 \item Prove in steps by rewriting each term with the goal of $Todo \cup Done$
 \item If any rewrite to \texttt{True}, the ``obvious term'' rule will rewrite the entire clause to \texttt{True}.
\end{itemize}

}

\section{Building a Sound Theorem Prover}

\begin{frame}[fragile]
\frametitle{Gadget Glue}
\begin{itemize}
 \item The key to proving all these gizmos sound is to do so in stages. 
 \item Milawa has an extra 10 layers on top of its core.
 \item Each layer admits hundreds to thousands of theorems.
 \item Some proofs get up to the order of Gigaconses.
\pause
 \item New justification steps are written in a proof building function.
\pause
 \item For lemmas, LISP Macros make proofs look very nice.
{\tiny
\begin{lstlisting}
 (defderiv clause.aux-split-twiddle-lemma-1
  :derive (v (v (v B C) (v A (v B C))) A)
  :from   ((proof x (v (v A C) (v B C))))
  :proof  (@derive
           ((v (v A C) (v B C))               (@given x))
           ((v (v B C) (v A C))               (build.commute-or @-))
           ((v A (v (v B C) (v A C)))         (build.expansion (@formula A) @-))
           ((v (v A (v B C)) (v A C))         (build.associativity @-))
           ((v (v (v A (v B C)) A) C)         (build.associativity @-))
           ((v C (v (v A (v B C)) A))         (build.commute-or @-))
           ((v B (v C (v (v A (v B C)) A)))   (build.expansion (@formula B) @-))
           ((v (v B C) (v (v A (v B C)) A))   (build.associativity @-))
           ((v (v (v B C) (v A (v B C))) A)   (build.associativity @-))))
\end{lstlisting}}
\end{itemize}
\end{frame}

\frame{
\frametitle{The Layers of Milawa}
\begin{description}
 \item[Layer 1] Core
 \item[Layer 2] Associativity of $\vee$, Modus ponens, clausifier.
 \item[Layer 3] Modus ponens list, generic subset
 \item[Layer 4] Groundwork. Obvious term elimination, substitute iff into literal, etc.
 \item[Layer 5] Equivalence traces, updating clauses
 \item[Layer 6] Factoring and splitting
 \item[Layer 7] Split tactics
 \item[Layer 8] Rewrite traces
 \item[Layer 9] Unconditional Rewriting, worlds
 \item[Layer 10] Conditional Rewriting, hypboxes
 \item[Layer 11] Tactics
\end{description}

}

\frame{
\frametitle{A Little Bit on Tactics}
\begin{itemize}
 \item Like LCF-style systems, a tactic is the pair of functions that apply and justify the tactic.
\pause
 \item First order $\Rightarrow$ cannot compose tactics dynamically. 
  \begin{itemize}
   \item Remember which tactics have been applied.
   \item Pass along non-proof information from tactics.
  \end{itemize}
\pause
 \item Tactics are implemented by creating ``proof skeletons'' that imply proof. These consist of:
 \begin{itemize}
  \item \emph{Goals} - list of clauses left to prove.
  \item \emph{Tacname} - name of the tactic used to produce above goals 
  \item \emph{Extras} - information the justifier may need.
  \item \emph{History} - the proof skeleton this to which tactic was applied.
 \end{itemize}
\end{itemize}
}

\frame{
\frametitle{Fleshing Out Proof Skeletons}
\begin{itemize}
 \item The justification function (verifier) constructs a proof from
 \begin{itemize}
  \item The skeleton produced by the tactic, $s$.
  \item Proofs of $s.Goals$
 \end{itemize}
 \item The proof it produces is a proof of $s.History.Goals$, recursively constructed, assuming $s.Goals$.
 \item Note: final skeleton has an empty goal clause.
\end{itemize}
}

\begin{frame}[fragile]
\frametitle{A sample tactic (case splitting)}
{\scriptsize
\begin{lstlisting}
(pequal* 
  (tactic.split-all-tac liftp llimit slimit skelly)
    (let ((goals (tactic.skeleton->goals skelly)))
      (if (not (consp goals))
          ;; fail: no clauses to prove
          nil
        (let* ((split
                 (clause.split-list liftp llimit
                                    slimit goals))
               (split-lens (strip-lens (cdr split)))
               (new-goals (simple-flatten (cdr split))))
             (if (not (car split))
                 ;; fail: no progress was made
                 nil
               (tactic.extend-skeleton new-goals
                                       'split-all
                                       (list liftp llimit slimit
                                             split-lens)
                                       skelly))))))
\end{lstlisting}
}
\end{frame}

\begin{frame}[fragile]
\frametitle{A sample tactic (case splitting) verifier}
{\scriptsize
\begin{lstlisting}
(pequal* 
  (tactic.split-all-compile x proofs)
  (let* ((history (tactic.skeleton->history x))
         (orig-goals (tactic.skeleton->goals history))
         (extras (tactic.skeleton->extras x))
         (liftp (first extras))
         (llimit (second extras))
         (slimit (third extras))
         (lens (fourth extras))
         (part-proofs (partition lens proofs)))
      (clause.split-list-bldr liftp llimit slimit
                              orig-goals part-proofs)))
\end{lstlisting}
}
\end{frame}

\frame{
\frametitle{Questions?}
Thank you.
}

\end{document}


